// -*- ASCII:EDT -*-
// Transform a Rnw file into a LaTeX file with Sweave (or cacheSweave) and
// and then pdfLaTeX it ...
// author: Gilbert Ritschard (gilbert.ritschard at unige.ch)


// arguments:
// Reg 6 : if set to "pdf", pdfLaTeX is done directly after Sweave when it ends without error
//         if set to "dvi"  LaTeXify is done directly
//         if set to "R" Stangle is run instead of Sweave
//         if set to "md" knitr markdown is used
// Reg 7 : should contain the path to the bin directory of your R implementation
// Reg 8 : if set to "cache", cacheSweave is called instead of Sweave
// Reg 9 : if set to "open" r Gui is started
//

//
// Modified by A. Blejec (andrej.blejec at nib.si) for Sweave Beamer (26. 4. 2010)
// Reg 9 : "S" for beamer slides, "A" for Beamer article ( file %nS.tex or %nA.tex needed )
// Reg 8 : stores name of main .Rnw file for reopening
// Rgui is opened with working directory %!p\r or %!p\wd if they exist
//
// From R 2.13.1 bin directory changed to /bin/i386
// check the Rgui start batch for correct command path
//
// 15. 3. 2012 modification for knitr
//
// 24. 4. 2012 Run patchSynctex.R for reverse lookup
//
// 07. 02. 2014 reg(4) added to tangle R call
//
// 11. 02. 2014 markdown option is added


Requires(20070913); // Requires this build of WinEdt to work properly
SetOK(1);  // Just in case ...
SetErrorFlag(0);
SaveAllDoc;
// First we retrieve the R install path
LetReg(5,"Install path");
// Prompt("%!5",0,1);
// ne vem zakaj spodnji ukaz vidi pravzaprav 'HKEY_LOCAL_MACHINE\SOFTWARE\Wow6432Node\R-core\R' ??? AntiVirus??
LetReg(5,|%@('HKEY_LOCAL_MACHINE','Software\Wow6432Node\R-core\R','InstallPath')|);
 Prompt("%!5",0,1);
  //// The path to the R binaries can be set before calling the macro (in the menu setup)
  //// for example: LetReg(7,"c:\Program Files\R\R-2.7.1\bin\")
// IfIStr("%!7","","=","LetReg(7,'%!5\bin\i386\');");
IfIStr("%!7","","=","LetReg(7,'%!5\bin\x64\');"); // for 64 bit version

 Prompt("%!7",0,1);

// Open RGui in subdirectory /R, /wd or in the main if it doesn't exists
// Modified by A. Blejec 25. 4. 2010
IfIStr("%!9","open","=","Relax","JMP('runR')");
    IfFileExists('%!p\r\',"LetReg(9,'%!p\r')","LetReg(9,'%p')");
    IfFileExists('%!p\wd\',"LetReg(9,'%!p\wd')","Relax");
    IfFileExists('%!7RGui.exe',"Run('%!7RGui.exe','%!9')","Prompt('%!7RGui.exe does not exist / Set Reg7 in Sweave.edt and check Rguistart.bat',0,1)");
JMP('exit');

:runR:
//  ResetConsole;
//  ShowConsole(1);
//  FocusConsole(1);
  CloseFile("%b\_Run.log",1);
  CloseFile("%b\_Err.log",1);
  IfFileExists("%b\_Err.log",!"DeleteFile('%b\_Err.log')");
  IfFileExists("%b\_Run.log",!"DeleteFile('%b\_Run.log')");



SetOK(1); // Just in case

  IfIStr("%!6","R","=","JMP('tangle');");
  IfIStr("%!6","md","<>","JMP('nonRmd');");
  IfIStr("%t","Rmd","<>","Exe('%b\macros\Sweave\TeX2Rmd.edt');");

:nonRmd:
  // Set LetReg(8,"cache") for cacheSweave

  IfFileExists('%b\Contrib\R-Sweave\%!8Sweave.R',"",>
       !"WrL('%%%%%% => %b\Contrib\R-Sweave\%!8Sweave.R not found <= ');");


  CloseFile('%p\%n.tex');
//  Open("%b\_Run.log");
//  CloseFile(%b\_Run.log");
//  Prompt("%!4");
  WinExe('','"%!7Rscript" "%b\Contrib\R-Sweave\%!8Sweave.R" "%n%t" "%!4" "%n" "%!6" "%b/Contrib/R-Sweave/"','%p',>
         'Sweave ...',0110,0,"", "", "%b\_Err.log", 110);
//         'Sweave ...',1110,0,"", "%b\_Run.log", "%b\_Err.log", 111);
  GetExitCode( 4 );
//  IfFileExists('%p\%n.tex',"Open('%p\%n.tex');");

//  Prompt("%!4");
  IfNum("%!4",0,"=","JMP('next');","JMP('exit');");
//-nolog   Open("%b\_Err.log");
//-nolog  SearchReset;
//-nolog  SetSearchCurrentDoc;
//-nolog  SetSearchEntire(1);
//-nolog  SetSearchForward(1);
//-nolog  SetSearchCaseSensitive(1);
//-nolog  //SetSearchInline(1);
//-nolog   //SetFindStr("You can now run LaTeX on"); // safer, but works only in English
//-nolog   SetFindStr("Error");
//-nolog   IfFound("Open('%b\_Run.log');Open('%b\_Err.log');JMP('exit');","JMP('next');");
:next:

 SearchReset;

  CloseFile("%b\_Err.log");
//  Reg 9: S and A for Beamer slide and article modes , "" if ordinary mode
//  Prompt("Register 9: %!9");
  LetReg(8,"%N.Rnw")  // save base Rnw file name AB
//  Prompt("%!F %F %f");
  IfIStr("%!6","md","=","LetReg(6,'MD');");
  IfIStr("%!6","MD","=","JMP('markdown');");
  IfIStr("%!6","HTML","=","JMP('markdown');");
  Open('%p\%N%!9.tex');
//  Prompt("Register 6: %!6");
// If Reg 6 is set to "dvi", TeXify %N,
// if set to "pdf" pdfTeXify,
// if set to "md" knitr markdown is executed
// if empty or "Stop" just set focus on the generated LaTeX file.
// "LetReg(6,'xxx');" should be set before launching the macro.

  IfIStr("%!6","","=","LetReg(6,'Stop');");
  IfIStr("%!6","pdf","=","LetReg(6,'PDF');");
  IfIStr("%!6","Direct","=","LetReg(6,'PDF');"); // just for backward compatibility
  IfIStr("%!6","dvi","=","LetReg(6,'');");


  IfIStr("%!6","Stop","=",>
     "Open('%p\%N.tex');",>
     "Exe('%%b\Exec\MikTeX\%!6TeXify.edt');CloseFile('%p\%N.tex');Open('%p\%!8');");  // open base Rnw file
 //     "Exe('%%B\Exec\TeX\%!6LaTeX.edt');Open('%p\%!8');");  // open base Rnw file - another option, not needed
 // run patchSweave
     WinExe('','"%!7Rscript" "%b\Contrib\R-Sweave\patchSynctex.R" "%N.Rnw"  >
         "%!4"','%p', 'patchSynctex ...',0110,0,"", "", "%b\_Err.log", 110);
  JMP('exit');

:markdown

  JMP('exit');

:tangle: ==========================

  LetReg(6,'%n.R');
  IfFileExists('%b\Contrib\R-Sweave\Stangle.R',"",>
       !"WrL('%%%%%% => %b\Contrib\R-Sweave\Stangle.R not found <= ');");

  CloseFile('%p\%n.R');
  WinExe('','"%!7Rscript" "%b\Contrib\R-Sweave\Stangle.R" "%n.Rnw"  "%!4"','%p',>
         'tangle ...',1110,0,"", "%b\_Run.log", "%b\_Err.log", 111);

  IfFileExists('%p\%n.R',"Open('%p\%n.R');");

  Open("%b\_Run.log");
 SearchReset;
 SetSearchCurrentDoc;
 SetSearchEntire(1);
 SetSearchForward(1);
 SetSearchCaseSensitive(1);
 //SetSearchInline(1);
  //SetFindStr("You can now run LaTeX on"); // safer, but works only in English
  SetFindStr("%!6");
  IfFound("CloseFile('%b\_Run.log');JMP('exit');","Open('%b\_Err.log');JMP('exit');");


:exit: ============================
  LetReg(4,'');
  LetReg(5,"");
  LetReg(6,"");
  LetReg(7,"");
  LetReg(8,"");
  LetReg(9,"");

  SetOK(1);
End;
